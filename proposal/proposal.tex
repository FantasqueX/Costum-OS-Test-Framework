\documentclass[UTF8]{ctexbeamer}
\usepackage[T1]{fontenc}

\usetheme[compress]{Berlin}
\setbeamertemplate{page number in head/foot}[framenumber]
%Information to be included in the title page:
\title{Rust 内核的模糊测试研究}
\subtitle{毕业设计开题报告}
\author{任乐图}
\institute{清华大学计算机科学与技术系}
\date{\today}

\begin{document}

\begin{frame}
    \titlepage
\end{frame}

\begin{frame}
    \frametitle{Outline}
    \tableofcontents
\end{frame}

\section{研究背景}
\begin{frame}
    \frametitle{研究背景}
    对于 C 编写 Linux 内核,测试的主要手段是 Syzkaller,进行覆盖率指导的模糊测试。而对于使用 Rust 编写的内核则没有很好的测试方法。
\end{frame}

\begin{frame}
    \frametitle{Syzkaller}
    Syzkaller 是一款覆盖率指导的模糊测试工具,在提供系统调用描述以及内核覆盖率信息后,可以有效测试内核。目前最先进的内核模糊测试研究几乎都与 Syzkaller 有关,如受 Syzkaller 启发的 Healer\cite{healer},基于 Syzkaller 的 StateFuzz\cite{statefuzz},PrIntFuzz\cite{printfuzz} 等。
\end{frame}

\section{选题内容}
\begin{frame}
    \frametitle{选题内容}
    测试功能相比 Linux 简化的、使用 Rust 编写的内核,尽可能多地发现 Bug。

    Bug 有两种:
    \begin{itemize}
        \item 内核崩溃;
        \item 内核的行为与标准(Linux)不一致。
    \end{itemize}
\end{frame}

\begin{frame}
    \frametitle{实验设计}
    \begin{itemize}
        \item 假设待测试的内核与 Linux 在输入相同时,代码覆盖率相同,则先用 Syzkaller 测试 Linux,得到测试数据。使待测试内核也运行同样的数据,比较两边的结果。
        \item 使 Syzkaller 直接测试待测试内核。主要的困难是待测试内核可能没有 KCov,以及 Rust 得到覆盖率的方法与 C 的有区别。
    \end{itemize}
\end{frame}

\section{进度安排}
\begin{frame}
    \begin{itemize}
        \item 1月12日,毕设开题。
        \item 1月12日至2月5日,熟悉 Syzkaller,使用 Syzkaller 测试 Linux。
        \item 2月5日至中期,使用 Linux 的测试数据测试待测试内核。
        \item 中期至结题,直接使用 Syzkaller 测试待测试内核。\cite{healer}
    \end{itemize}
\end{frame}

\section{参考文献}
\begin{frame}
    \tiny\bibliographystyle{acm}
    \bibliography{ref}
\end{frame}

\end{document}

